\thispagestyle{empty}

\begin{center}
	{\large\bfseries LATEN \\ Larva Agent Telegram Notifier }\\
\end{center}
\begin{center}
	Francisco Domínguez Lorente\\
\end{center}

%\vspace{0.7cm}

\vspace{0.5cm}
\noindent{\textbf{Palabras clave}: \textit{LATEN}, \textit{red de agentes}, \textit{ACL}, \textit{bot}, \textit{Telegram}}
\vspace{0.7cm}

\noindent{\textbf{Resumen}}\\
	\textbf{LATEN}, acrónimo de \textbf{L}arva \textbf{A}gent \textbf{TE}legram \textbf{N}otifier es un agente diseñado para facilitar y mejorar la retroalimentación
	de los ejercicios prácticos y en general, de los contenidos impartidos en la asignatura \textit{Desarrollo Basado en Agentes} del grado en Ingeniería Informática.\\

	El propósito de dicho agente será pertenecer a la red existente de agentes de dicha asignatura, que se comunican entre ellos a través del estándar ACL \textit{(Agent Communication Language)},
	para proveer sobre todo al estudiante de una mejor retroalimentación e información general sobre su progreso en los ejercicios prácticos de la asignatura.\\

	LATEN es un agente nuevo que se apoya en el intercambio de mensajes para realizar sus funciones, al igual que el resto de agentes de la red. Estas funciones se realizan a través de un
	bot de Telegram, al cual se le pueden enviar ciertos comandos establecidos para que mediante un mensaje, este devuelva la información requerida. Entre otras funciones, se encuentran, por ejemplo:
    
	\begin{itemize}
		\item Recepción de notificaciones de las ejecuciones de los ejercicios prácticos
		\item Información sobre el progreso actual de los ejercicios prácticos
		\item Comprobar el estado de los agentes del sistema
	\end{itemize}
    
	Para realizar algunas de las tareas, se utiliza el propio agente LATEN y para otras se añaden funcionalidades a los agentes existentes dedicados a la gestión de todos los eventos
	relacionados al bot de Telegram.\\

	Adicionalmente, también se provee al profesor de ciertas funcionalidades para también facilitar las labores de docencia y de gestión de los propios ejercicios prácticos que
	se desarrollan a lo largo de la asignatura.

\cleardoublepage

\begin{center}
	{\large\bfseries LATEN \\ Larva Agent Telegram Notifier }\\
\end{center}
\begin{center}
	Francisco Domínguez Lorente\\
\end{center}
\vspace{0.5cm}
\noindent{\textbf{Keywords}: \textit{LATEN}, \textit{agent network}, \textit{ACL}, \textit{bot}, \textit{Telegram}}
\vspace{0.7cm}

\noindent{\textbf{Abstract}}\\
    \textbf{LATEN}, acronym of \textbf{L}arva \textbf{A}gent \textbf{TE}legram \textbf{N}otifier is an agent designed to facilitate and improve the feedback of the assignments and, in general, of the imparted contents in the \textit{Agent Based Development} subject of the Computer Engineering degree.\\
    
    The purpose of said agent is to belong to the existing agent network of the subject, that communicate among them through the ACL \textit{(Agent Communication Language)} standard, to provide specially the students of a better feedback and general information about their progress in the assignments of the subject.\\
    
    LATEN is a new agent that relies on the message exchange to perform its functions, like the rest of the agents in the network. This functions are performed through a Telegram bot, to which it can be send different, already established commands to, throughout a message, return the required information. For example, within those functions, we have:
    
    \begin{itemize}
		\item Receipt of notifications of the execution of the assignments
		\item Information about the current progress of the assignments
		\item Check the status of the agents in the system
	\end{itemize}
	
	The LATEN agents itself is being used to perform some of the tasks, but for other tasks, the existing agents dedicated to the management of all the events related to the Telegram bot are used.\\
	
	Additionally, certain features are provided to the teacher to ease the teaching labors and management of the assignments that take place throughout the subject.

\cleardoublepage

\thispagestyle{empty}

\noindent\rule[-1ex]{\textwidth}{2pt}\\[4.5ex]

D. \textbf{Luis Castillo Vidal}, Profesor(a) del Departamento de Ciencias de la Computación e Inteligencia Artificial

\vspace{0.5cm}

\textbf{Informo:}

\vspace{0.5cm}

Que el presente trabajo, titulado \textit{\textbf{LATEN}},
ha sido realizado bajo mi supervisión por \textbf{Francisco Domínguez Lorente}, y autorizo la defensa de dicho trabajo ante el tribunal
que corresponda.

\vspace{0.5cm}

Y para que conste, expiden y firman el presente informe en Granada a Septiembre de 2021.

\vspace{1cm}

\textbf{El/la director(a)/es: }

\vspace{5cm}

\noindent \textbf{Luis Castillo Vidal}

\chapter*{Agradecimientos}




