\chapter{Descripción del problema}

\section{Antecedentes}

Este Trabajo de Fin de Grado surge a raíz de la asignatura Desarrollo Basado en Agentes del Grado en Ingeniería Informática. Dicha asignatura constaba de una parte teórica y una parte práctica, pero la mayor parte de la calificación final de la asignatura iba a ser obtenida gracias a esa parte práctica debido a su gran carga lectiva.\\

Esta parte práctica se subdivide en varias prácticas, que consisten en hacer uso de las tecnologías mencionadas en el \autoref{chap:introduccion} de este documento, para lograr que una red de agentes, concretamente una flota de drones consiga ''rescatar'' a varias personas que se encuentran distribuidas en un mapa y cuya posición se desconoce a priori. Para ello, estos drones hacen uso de ciertos sensores que pueden adquirir mediante comunicaciones con otros agentes.\\

Para controlar el progreso en cada una de las prácticas y para también obtener retroalimentación sobre las ejecuciones que lancemos de nuestros programas hacia el entorno, se habilitan dos plataformas independientes entre sí, pero que se pueden usar de forma conjunta para obtener un balance general de las prácticas de la asignatura:

\begin{itemize}
	\item \textbf{Una página web} enlazada al ecosistema LARVA en la cual se muestran los hitos, logros, comopetencias y ejercicios que debe realizar el alumno, además del progreso actual en cada uno de esos elementos.
	\item \textbf{Un bot de telegram} que aporta funcionalidades en tiempo real sobre la ejecución de los problemas de las prácticas, así como otro tipo de retroalimentación, como por ejemplo puede ser el estado de los agentes implicados en el sistema.
\end{itemize}

\section{Motivación}

Tras haber cursado la asignatura Desarrollo Basado en Agentes y haber hecho uso de las propias herramientas que nos proporcionó el profesor, era evidente que ambas plataformas podrían ser mejoradas para mejorar la experiencia de uso sobre todo de los propios alumnos. Durante todo el curso, tanto mis compañeros de clase como yo personalmente hemos comentado cómo se podrían mejorar estas plataformas de manera que la retroalimentación que obtuviésemos fuese lo más completa posible.\\

En general, el consenso con los compañeros era que la asignatura podría haber sido más llevadera si estas herramientas gozasen de más características y retroalimentación, para así saber en qué estamos fallando a la hora de realizar nuestras prácticas y que por tanto el uso del tiempo fuese más eficiente.

\section{Objetivos}

El objetivo propiamente de este Trabajo de Fin de Grado no es más que mejorar el existente bot de Telegram, desarrollado con anterioridad por el profesor de la asignatura D. Luis Castillo Vidal, dotándolo de más funcionalidades que ayuden al estudiante tanto en tiempo real, como activamente a través del uso de comandos preestablecidos en el bot que le aporten la retroalimentación necesaria para aprovechar de manera más eficiente su tiempo.

\section{Descripción del problema}

La problemática radica en la comprensión y adaptación del código existente para así incorporar nuevas funcionalidades que se detallarán en capítulos próximos. El uso de las tecnologías mencionadas anteriormente, como por ejemplo JADE, a un nivel más detallado a diferencia de como lo hemos estado tratando como alumnos en la asignatura, hace que este proyecto requiera de un esfuerzo de investigación importante.\\

Además, como es un proyecto existente, también es necesario entender el código existente, así como la estructura de la base de datos actual y la toma de decisiones que se llevaron a cabo a la hora de escribir dicho código y de definir la estructura de la base de datos, lo cual también supone un esfuerzo adicional.