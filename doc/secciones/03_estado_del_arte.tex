\chapter{Estado del arte}

Durante el desarrollo de las prácticas de la asignatura Desarrollo Basado en Agentes, se nos proporcionaron todos los materiales necesarios para proceder a la creación del agente que realice las tareas relativas a la práctica correspondiente. Estos materiales son los que se han explicado previamente en la sección \ref{chap:introduccion} de este documento, más concretamente los materiales relativos a JADE y a LARVA.\\

Con respecto a la parte práctica de dicha asignatura, como ya se ha comentado anteriormente, existen una serie de desafíos que se apoyarán en esos materiales ya implementados por el tutor de la asignatura.\\

El estado en el cual estaba el ecosistema al empezar a desarrollar este proyecto era bastante consistente, teniendo en cuenta que el framework JADE era totalmente nuevo en la asignatura y que el profesor de la misma había implementado por primera vez el curso pasado. Antes de utilizar JADE, se utilizaba Magentix, el cual es otra plataforma que soporta e implementa sistemas multiagentes.\\

Al comienzo de la asignatura, se dieron lugar ciertos desafíos individuales que se utilizaron para la familiarización tanto con JADE, como con LARVA, así como con los mensajes ACL. Inicialmente, debido a que la plataforma era prácticamente nueva, nos encontramos con ciertas casuísticas que no habían sido contempladas por el profesor y que causaban que la plataforma no funcionase correctamente. Para ser un poco más concretos, en algunas situaciones no se consideraban algunos escenarios límite que posteriormente resultaron bastante frecuentes entre el alumnado que intentaba entender cómo funcionaba todo el entorno a base de \textit{ensayo-error}.\\

Estas casuísticas fueron solventadas sobre la marcha por el propio tutor, que en varias ocasiones nos avisó de que la plataforma no estaría disponible por el mantenimiento de la misma y para implantar las soluciones a dichos problemas. Eventualmente a lo largo del curso se solventaron todos los problemas de funcionamiento, y de cara a los últimos desafíos, a los que eran grupales, el ecosistema gozaba de una alta estabilidad, fiabilidad y tolerancia a errores, lo cual facilitó en gran medida la realización de las prácticas.\\

Adicionalmente, y de la base que parte este proyecto, es del ya mencionado bot de Telegram, que ya existía y que también fue implementado por el profesor de la asignatura el curso pasado, junto con la nueva plataforma de agentes JADE. Este bot de Telegram en un inicio también gozaba de bastante estabilidad y de algunas funcionalidades implementadas por el tutor durante su desarrollo, pero en adición a lo comentado anteriormente, sería una herramienta mucho más potente y que nos otorgaría a alumnado de mayor independencia si estuviese más completa en cuando a funcionalidad se refiere y si la retroalimentación que nos proporcionase también fuese mejor y más adaptada a nuestras necesidades.\\

Como fue un sistema, tanto la plataforma JADE como el bot de Telegram, desarrollados en primera instancia por el profesor de la asignatura, no todas las opiniones y sugerencias que nosotros le transmitíamos no podían ser incorporadas, ya sea por falta de tiempo o por cuestiones de prioridad sobre otras tareas más críticas. Así pues, sobre todo la plataforma JADE fue objeto de constantes mejoras, pero el bot de Telegram, al no ser tan relevante para el correcto funcionamiento de las prácticas, fue relevado por el profesor a un segundo plano de prioridad. Es de ahí de donde surge este proyecto.\\

Por último, para comentar brevemente, con respecto a la base de datos existente, en varias conversaciones con el tutor de este proyecto hemos llegado al acuerdo de que la estructura existente, que será comentada más en detalle posteriormente, la que inicialmente se pensó para el desarrollo de la asignatura quizás no sea la más apropiada por su alta complejidad a la hora de realizar consultas sobre la misma. No obstante, a pesar de estar de acuerdo en que quizás sería conveniente un cambio en la estructura, por falta de tiempo en la planificación no se ha considerado su puesta en marcha.