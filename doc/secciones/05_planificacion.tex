\chapter{Planificación}

\section{Metodología utilizada}

Incluso desde el inicio de la asignatura Desarrollo Basado en Agentes, el profesor de dicha asignatura D. Luis Castillo Vidal nos exigió el uso de metodologías ágiles para el desarrollo de cada una de las prácticas, que se desarrollaban por grupos generalmente.\\

Para controlar todo el progreso de cada uno de los miembros, el profesor nos facilitó ciertas hojas de cálculo, una para cada grupo, donde cada grupo se encargaría de definir los siguientes elementos:

\begin{itemize}
	\item Cada una de las tareas que se van a llevar a cabo a lo largo de la práctica actual.
	\item Estimación de cuántas horas cada alumno se compromete a dedicar en la práctica actual.
	\item Estimación de cuántas horas cada tarea (o historia) puede ocupar. Es necesario que la suma de esta estimación coincida con la suma de la estimación de horas de cada alumno del grupo.
	\item Registrar las horas dedicadas a cada historia por parte de cada alumno.
	\item Definir, para cada práctica, un miembro del grupo que ejercerá como líder. Es quien debería de mantener las comunicaciones con el profesor.
\end{itemize}

De igual forma, para este proyecto se ha seguido una estrategia similar, que podría definir informalmente como \textit{pseudo-SCRUM}. En los primeros días de definición del proyecto, una vez estaban las bases consensuadas con mi tutor D. Luis Castillo Vidal, también hicimos una sesión de inicio de la metodología, en la que esencialmente llevamos a cabo algunas de las tareas mencionadas anteriormente.\\

Más concretamente, definimos la fecha de comienzo y fin del proyecto, y también definimos una serie de \textit{sprints}, con una cantidad fija de días cada uno tras los cuales organizaríamos una sesión de control para evaluar cómo iba avanzando el proyecto. Además, se consensuaron todas las historias a realizar durante todo el proyecto y se hizo la estimación general en horas.

\section{Temporización}

Antes de comentar en profundidad la temporización, es destacable comentar que inicialmente se realizó una fase de análisis en febrero, por la cual se estipuló dar comienzo al proyecto el día 3 de Marzo del presente año y marcarlo como finalizado el día 15 de julio también del presente año, dividido en siete \textit{sprints} de 21 días cada uno a excepción del último. Debido a ciertos problemas, nos vimos obligados a posponer este proyecto y a realizar una nueva fase de replanificación en junio.\\

Durante la fase de análisis del proyecto, se llegó a un consenso con el tutor de establecer como fecha de comienzo el día 7 de junio de 2021 y fecha de finalización del mismo el 14 de septiembre del mismo año. Esta vez, se estipularon diez \textit{sprints} de diez días cada uno.\\

El proyecto se temporizó en 100 unidades de tiempo a completar, donde cada unidad de tiempo equivaldría entre tres y cuatro horas de trabajo real del alumno.\\

También se reestructuraron, en consenso con el tutor, las historias que previamente habíamos considerado para el proyecto en febrero para así ajustar de nuevo el tiempo que teníamos disponible y también teniendo en cuenta el trabajo previo que realicé anteriormente.\\

Surgieron así, 13 tareas o historias que deberían ser completadas a lo largo del periodo planificado anteriormente. En la tabla REF, se muestra un resumen de cada historia junto con el tiempo consensuado con el tutor.

\section{Seguimiento del desarrollo}
