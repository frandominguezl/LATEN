\chapter{Planificación}

\section{Metodología utilizada}

Incluso desde el inicio de la asignatura Desarrollo Basado en Agentes, el profesor de dicha asignatura D. Luis Castillo Vidal nos exigió el uso de metodologías ágiles para el desarrollo de cada una de las prácticas, que se desarrollaban por grupos generalmente.\\

Para controlar todo el progreso de cada uno de los miembros, el profesor nos facilitó ciertas hojas de cálculo, una para cada grupo, donde cada grupo se encargaría de definir los siguientes elementos:

\begin{itemize}
	\item Cada una de las tareas que se van a llevar a cabo a lo largo de la práctica actual.
	\item Estimación de cuántas horas cada alumno se compromete a dedicar en la práctica actual.
	\item Estimación de cuántas horas cada tarea (o historia) puede ocupar. Es necesario que la suma de esta estimación coincida con la suma de la estimación de horas de cada alumno del grupo.
	\item Registrar las horas dedicadas a cada historia por parte de cada alumno.
	\item Definir, para cada práctica, un miembro del grupo que ejercerá como líder. Es quien debería de mantener las comunicaciones con el profesor.
\end{itemize}

De igual forma, para este proyecto se ha seguido una estrategia similar, que podría definir informalmente como \textit{pseudo-SCRUM}. En los primeros días de definición del proyecto, una vez estaban las bases consensuadas con mi tutor D. Luis Castillo Vidal, también hicimos una sesión de inicio de la metodología, en la que esencialmente llevamos a cabo algunas de las tareas mencionadas anteriormente.\\

Más concretamente, definimos la fecha de comienzo y fin del proyecto, y también definimos una serie de \textit{sprints}, con una cantidad fija de días cada uno tras los cuales organizaríamos una sesión de control para evaluar cómo iba avanzando el proyecto. Además, se consensuaron todas las historias a realizar durante todo el proyecto y se hizo la estimación general en horas.

\section{Temporización}

Antes de comentar en profundidad la temporización, es destacable comentar que inicialmente se realizó una fase de análisis en febrero, por la cual se estipuló dar comienzo al proyecto el día 3 de Marzo del presente año y marcarlo como finalizado el día 15 de julio también del presente año, dividido en siete \textit{sprints} de 21 días cada uno a excepción del último. Debido a ciertos problemas, nos vimos obligados a posponer este proyecto y a realizar una nueva fase de replanificación en junio.\\

Durante la fase de análisis del proyecto, se llegó a un consenso con el tutor de establecer como fecha de comienzo el día 7 de junio de 2021 y fecha de finalización del mismo el 14 de septiembre del mismo año. Esta vez, se estipularon diez \textit{sprints} de diez días cada uno.\\

El proyecto se temporizó en 100 unidades de tiempo a completar, donde cada unidad de tiempo equivaldría entre tres y cuatro horas de trabajo real del alumno.\\

También se reestructuraron, en consenso con el tutor, las historias que previamente habíamos considerado para el proyecto en febrero para así ajustar de nuevo el tiempo que teníamos disponible y también teniendo en cuenta el trabajo previo que realicé anteriormente.\\

Surgieron así, 13 tareas o historias que deberían ser completadas a lo largo del periodo planificado anteriormente. En la tabla \ref{tab:historias}, se muestra un resumen de cada historia junto con el tiempo consensuado con el tutor.\\

En dicha tabla no obstante, aparecen 14 historias, y si además sumamos el tiempo estimado para cada una de estas historias, obtendríamos que el total de unidades de tiempo planificadas es de 112 en lugar de 100. Esto se debe a que en febrero, en la fase inicial de análisis, a pesar de que no se puso seguir adelante con el proyecto, yo como estudiante llevé a cabo algunas tareas de investigación y comprensión que formaban parte de la planificación original, y que no se incluyeron de nuevo en la planificación ya que estaban completadas. Es por eso que se incluyeron de manera adicional indicando el número de horas que ya se había trabajado en el proyecto.

% Please add the following required packages to your document preamble:
% \usepackage{graphicx}
\begin{table}[]
\centering
\resizebox{\textwidth}{!}{%
\begin{tabular}{|l|l|c|}
\hline
\textbf{Historia}                                                                          & \textbf{Descripción}                                                                                                                                                                                                                                                                                                                                                                                                                                & \textbf{Tiempo Estimado} \\ \hline
Completar ciclo de simulador                                                               & \begin{tabular}[c]{@{}l@{}}Leer, interpretar y ejecutar archivos de registro completos. \\ Desde que se analiza cada línea del archivo, \\ hasta que se hace la comunicación con el agente de \\ Telegram.\end{tabular}                                                                                                                                                                                                                             & 7                        \\ \hline
Simular varios receptores                                                                  & \begin{tabular}[c]{@{}l@{}}Probar la recepción de mensajes de Telegram a \\ diferentes usuarios\end{tabular}                                                                                                                                                                                                                                                                                                                                        & 5                        \\ \hline
Procesar entradas de chat \#1                                                              & Identificación de cada usuario usando su cardID                                                                                                                                                                                                                                                                                                                                                                                                     & 5                        \\ \hline
Procesar entradas de chat \#2                                                              & \begin{tabular}[c]{@{}l@{}}Definir opciones de suscripción para notificaciones,\\ siendo capaz de definir:\\ - Recibir o no notificaciones\\ - Si se reciben notificaciones: recibir todas las \\ notificaciones, recibir solo las notificaciones\\ imprescindibles (mensajes de error)\\  o recibir notificaciones de tipo ACL\end{tabular}                                                                                                        & 12                       \\ \hline
Ofrecer progresos individuales                                                             & \begin{tabular}[c]{@{}l@{}}Perfil de progreso individual (o de grupo para aquellas \\ prácticas que sean en grupos)\end{tabular}                                                                                                                                                                                                                                                                                                                    & 7                        \\ \hline
Ofrecer progreso por problema                                                              & Perfil de progreso por un problema concreto                                                                                                                                                                                                                                                                                                                                                                                                         & 5                        \\ \hline
Ofrecer progreso por práctica                                                              & Perfil de progreso por una práctica concreta                                                                                                                                                                                                                                                                                                                                                                                                        & 7                        \\ \hline
Retroalimentaciones simples                                                                & - Fecha de entrega de una práctica concreta                                                                                                                                                                                                                                                                                                                                                                                                         & 6                        \\ \hline
\begin{tabular}[c]{@{}l@{}}Desplegar el sistema a un \\ entorno de producción\end{tabular} & \begin{tabular}[c]{@{}l@{}}Gestionar información en tiempo real, con conexión \\ directa a la base de datos y con agentes reales.\\ Verificar todas las historias previas.\end{tabular}                                                                                                                                                                                                                                                             & 15                       \\ \hline
Procesar entradas de chat \#3                                                              & \begin{tabular}[c]{@{}l@{}}Comprobar el estado de los agentes. Informar\\  al profesor a través de Telegram si uno de los \\ agentes no se encuentra disponible.\end{tabular}                                                                                                                                                                                                                                                                       & 5                        \\ \hline
Documentación                                                                              & Elaboración de la memoria                                                                                                                                                                                                                                                                                                                                                                                                                           & 16                       \\ \hline
Presentación                                                                               & Elaboración de la presentación para la defensa del proyecto                                                                                                                                                                                                                                                                                                                                                                                         & 10                       \\ \hline
Trabajo previo (desde febrero)                                                             & \begin{tabular}[c]{@{}l@{}}En resumen:\\ - Lectura de manuales de JADE\\ - Análisis y comprensión de proyectos relacionados con \\ \textit{behaviours} para entender mejor su funcionamiento\\ - Análisis y comprensión de la arquitectura LARVA\\ - Estudio de agentes y componentes concretos: ConfigFile, \\ Logger, ACLMessageQueue, ACLMSplitQueue,\\  Jerarquía de agentes...\\ - Estudio de la API de Telegram\end{tabular} & 12                       \\ \hline
\end{tabular}%
}
\caption{Backlog de historias estimadas para el proyecto. Cada unidad de tiempo corresponde a 3-4 horas de trabajo real del alumno.}
\label{tab:historias}
\end{table}

\section{Seguimiento del desarrollo}

El seguimiento de cada uno de los \textit{sprints} y de cada una de las tareas se realizó mediante una hoja de cálculo alojada en Google Drive, la cual fue puesta a puesto y configurada por el tutor de este proyecto, adaptándola a las necesidades concretas del mismo.\\

La finalidad de esta hoja de cálculo es controlar el número de horas que se dedican a cada una de las historias, en cada uno de los diez \textit{sprints} que se estipularon. De esta forma y gracias a las configuraciones realizadas por el tutor, es posible ver de una forma muy visual el progreso tanto del proyecto a nivel general como del \textit{sprint} que se desee, como se muestra en la figura \ref{img:sprint3}.\\

\begin{figure}[h]
\centering
\includegraphics[width=0.9\textwidth]{logos/sprint4.png}\\[1.4cm]
\caption{Progreso del \textit{sprint} número 3. La línea azul representa el progreso ideal, mientras que la línea roja es el progreso real del estudiante durante ese \textit{sprint}.}
\label{img:sprint3}
\end{figure}

Como estudiante, me gustaría aclarar que este procedimiento para llevar la cuenta de todas las horas dedicadas al proyecto resulta muy eficiente, ya que constantemente se obtiene retroalimentación real de cuánto va avanzando el proyecto, tanto a nivel general como a nivel concreto de historias.