\chapter{Conclusiones y trabajos futuros}

A pesar de que durante la realización de la asignatura Desarrollo Basado en Agentes se adquirieron los conocimientos necesarios para llevar a cabo todas las prácticas de dicha asignatura, este proyecto ha servido para profundizar en muchos de esos aspectos y a su vez descubrir e investigar más de cerca las capas inferiores del desarrollo de agentes que no estaban visibles durante la asignatura.\\

Más concretamente, se ha comprobado de cerca cómo funcionan los comportamientos \textit{(behaviours)} de los agentes y otras especificaciones técnicas de JADE, como por ejemplo las colas de mensajes. Se han seguido durante el desarrollo del proyecto los estándares oportunos estipulados por la FIPA y se han intentado seguir los mismos patrones de diseños usados anteriormente por el creador de dichos proyectos, el profesor de la asignatura y tutor de este proyecto D. Luis Castillo Vidal.\\

Con respecto a los trabajos futuros, se podrían considerar las siguientes mejoras:\\

\begin{itemize}
	\item Mejorar el esquema de la base de datos
	\item Implementar funcionalidades en LATEN que puedan ayudar al profesor de la asignatura con las labores docentes
	\item Ampliar las funcionalidades de LATEN para los estudiantes
	\item Complementar LATEN con otros agentes de la plataforma para obtener mejores retroalimentaciones
\end{itemize}

Algunas de ellas estaban fuera del objetivo de este proyecto y otras simplemente no han sido planificadas, ya que en el marco del proyecto se contemplaban tareas más importantes para el desarrollo del mismo.